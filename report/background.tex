Several tasks exist around the modelling of human motion and the given task usually dictates the shape of the model to a certain extent.

One such example is animating a human 3D mesh to conform with input from a game controller or equivalent keyboard input.
This task is of particular interest for the video game industry.
One such model is a phase-functioned neural network, wherein multiple networks are samples in a phase, as to mimic the repetitive nature of walking~\cite{holden2017}.
This approach can be extended to other tasks, such as playing basketball~\cite{starke2020}.
This approach values being able to run in real-time higher than animation quality.
This makes it ideal for video games, but less ideal for other applications where the animations are typically produced in advance and as such there is no requirement for real-time production.

Another example is motion prediction, where the model is given a sequence of human poses and asked to generate future human poses.
This is of particular interest for the autonomous car industry, but also has a potential impact for all industries that produce animations, which includes industries working with: movies, video games, entertainment, infotainment, education, etc.
As auto-completing an animation could potentially speed up the time it takes to produce a finished animation.
A common approach to this task is to use recurrent neural networks~\cite{hu2020predicting,jain2016structuralrnn}.
Generative adversarial networks (GANs) are commonly used in image based generative tasks, such as generating human faces~\cite{karras2019stylebased}.
GANs have been shown to also be effective for human motion prediction~\cite{ruiz2019human}.

In-painting of human motion can be considered a special case of motion prediction, where the model gets both the past and future motion, instead of only getting past motion.
This could be used to help speed up the animation of humanoids for video games, movies, etc.
It seems that data on this particular task is relatively sparse.
